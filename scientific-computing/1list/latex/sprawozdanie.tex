\documentclass[11pt, a4paper]{article}
\usepackage[utf8]{inputenc}

\title{Obliczenia naukowe\\Lista1}
\author{Stanisław Woźniak}
\date{}

\begin{document}
\maketitle


\section{Zadanie 1}
\subsection{Macheps - Epsilon Maszynowy}
\subsection{Eta}
\subsection{Max Float - Największa wartość}

\section{Zadanie 2 - Wzór na macheps}
\subsection{Problem}
Należało stwierdzić eksperymentalnie prawdziwość wzoru Kahana na epsilon maszynowy (macheps), który jest według niego opisany wzorem:
\[3*(\frac{4}{3} - 1) - 1\]
\subsection{Wyniki}
Porównanie wyników z poprawnym epsilonem maszynowym:
\begin{center}
\begin{tabular}{ c|c|c}
  & Wyliczony według wzoru & poprawny\\
  \hline
 Float16 & -0.000977 & 0.000977\\
 Float32 & 1.1920929e-7 & 1.1920929e-7\\
 Float64 & -2.220446049250313e-16 & 2.220446049250313e-16
\end{tabular}
\end{center}
\subsection{Wnioski}


\section{Zadanie 3 - Rozmieszczenie liczb zmiennopozycyjnych}
\subsection{Problem}
Należało sprawdzić eksperymentalnie własność liczb w arytmetyce Float64 w przedziale [1, 2], że każda z nich może być przedstawiona wzorem:
\[x =1 + k * \delta \] gdzie $k = 1, 2, ... 2^{52}-1$ oraz $\delta = 2^{-52}$
\subsection{Wynik}
Wyniki wychodzą zgodne dla podanego wzoru porównywanego z funkcją nextfloat. Przy porównaniu liczb w bitach możemy również zauwazyc, że wyniki są takie same w obu przypadkach.
\subsection{Wnioski}
Podany wzór jest poprawny, gdyż liczba $\delta$ jest reprezentowana jako jeden najmniej znaczący bit w arytmetyce Float64. Mnożąc po kolei przez każdą liczbę k uzyskujemy każdą możliwą reprezentację 52 najmniej znaczacych bitów, czyli mantysy.
\\
Natomiast liczby w przedzale [$\frac{1}{2}$, 1] są rozmieszczone z różnicą $\delta = 2^{-53}$, a w przedziale [2, 4] sa rozmieszczone z różnicą $\delta = 2^{-51}$, ponieważ liczba ostateczna jest w reprezentacji $mantysa*2^{cecha}$. 


\section{Zadanie 4}

\section{Zadanie 5}

\section{Zadanie 6}

\section{Zadanie 7}

\end{document}
