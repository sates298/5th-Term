\documentclass[12pt, a4paper]{article}
\usepackage[T1]{fontenc}
\usepackage[utf8]{inputenc}
\usepackage[MeX]{polski}
\usepackage[polish]{babel}
\usepackage{enumerate}

\title{Obliczenia naukowe\\Lista1}
\author{Stanisław Woźniak}
\date{}

\begin{document}
\maketitle


\section{Zadanie 1}
\subsection{Macheps - Epsilon Maszynowy}
\subsection{Eta}
\subsection{Max Float - Największa wartość}

\section{Zadanie 2 - Wzór na macheps}
\subsection{Problem}
Należało stwierdzić eksperymentalnie prawdziwość wzoru Kahana na epsilon maszynowy (macheps), który jest według niego opisany wzorem:
\[3*(\frac{4}{3} - 1) - 1\]
\subsection{Wyniki}
Porównanie wyników z poprawnym epsilonem maszynowym:
\begin{center}
\begin{tabular}{ c|c|c}
  & Wyliczony według wzoru & poprawny\\
  \hline
 Float16 & -0.000977 & 0.000977\\
 Float32 & $1.1920929 * 10^{-7}$ & $1.1920929 * 10^{-7}$\\
 Float64 & $-2.220446049250313 * 10^{-16}$ & $2.220446049250313 * 10^{-16}$
\end{tabular}
\end{center}
\subsection{Wnioski}


\section{Zadanie 3 - Rozmieszczenie liczb zmiennopozycyjnych}
\subsection{Problem}
Należało sprawdzić eksperymentalnie własność liczb w arytmetyce Float64 w przedziale [1, 2], że każda z nich może być przedstawiona wzorem:
\[x =1 + k * \delta \] gdzie $k = 1, 2, ... 2^{52}-1$ oraz $\delta = 2^{-52}$
\subsection{Wynik}
Wyniki wychodzą zgodne dla podanego wzoru porównywanego z funkcją nextfloat. Przy porównaniu liczb w bitach możemy również zauważyć, że wyniki są takie same w obu przypadkach.
\subsection{Wnioski}
Podany wzór jest poprawny, gdyż liczba $\delta$ jest reprezentowana jako jeden najmniej znaczący bit w arytmetyce Float64. Mnożąc po kolei przez każdą liczbę k uzyskujemy każdą możliwą reprezentację 52 najmniej znaczacych bitów, czyli mantysy.

Natomiast liczby w przedzale [$\frac{1}{2}$, 1] są rozmieszczone z różnicą $\delta = 2^{-53}$, a w przedziale [2, 4] sa rozmieszczone z różnicą $\delta = 2^{-51}$, ponieważ liczba ostateczna jest w reprezentacji $mantysa*2^{cecha}$. Wynikiem czego są różnice pomiędzy $\delta$ w przedziale [$\frac{1}{2}$, 1] a $\delta$ w przedziale [1, 2] lub [2, 4].
\section{Zadanie 4}
\subsection{Problem}
Należało znaleźć liczbę w arytmetyce Float64 z przedziału $1 < x < 2$, taką, że $x * \frac{1}{x} \neq 1$.
\subsection{Wynik}
\begin{center}
  \begin{tabular}{c|c}
    najmniejsza & największa\\
    \hline
    1.000000057228997 & 1.9999999850988384
  \end{tabular}
\end{center}
\subsection{Wnioski}


\section{Zadanie 5 - Iloczyn skalarny dwóch wektorów}
\subsection{Problem}
Przy pomocy 4 różnych algorytmów do policzenia iloczynu skalarnego, należało porównać wyniki do prawidłowej wartości: $ -1.00657107000000 * 10^{-11}$.

Dane wektory:
\[x = [2.718281828, -3.141592654, 1.414213562, 0.5772156649, 0.3010299957]\]
\[y = [1486.2497, 878366.9879, -22.37492, 4773714.647, 0.000185049]\]
\subsection{Algorytmy i ich wyniki}
\begin{enumerate}
  \item Algorytm liczący sumę kolejnych iloczynów wartości wektora
  \item Algorytm liczący sumę kolejnych iloczynów wartości wektora zaczynając od ostatniego
  \item Algorytm po wyliczeniu iloczynów dodaje wyniki dodatnie od największego do najmniejszego, następnie dodaje osobno wyniki ujemne od najmniejszego do największego. Ostatecznie obie składowe sumują się w końcowy wynik.
  \item Algorytm po wyliczeniu iloczynów dodaje wyniki dodatnie od najmniejszego do największego, następnie dodaje osobno wyniki ujemne od największego do najmniejszego. Ostatecznie obie składowe sumują się w końcowy wynik.
\end{enumerate}
Wyniki:
\begin{center}
  \begin{tabular}{c|c|c}
    algorytm & wynik dla Float32 & wynik dla Float64\\
    \hline
    pierwszy & -0.4999443 & $1.0251881368296672 * 10^{-10}$\\
    drugi & -0.4543457 & $-1.5643308870494366 * 10^{-10}$\\
    trzeci & -0.5 & 0.0\\
    czwarty & -0.5 & 0.0
  \end{tabular}
\end{center}
poprawna wartość: \[-1.00657107000000 * 10^{-11}\]
\subsection{Wnioski}

\section{Zadanie 6 - Porównanie dwóch takich samych funkcji}
\subsection{Problem}
Podane zostały dwie funkcje, które w praktyce są identyczne. W arytmetyce Float64 należało porównać wyniki każdej z nich dla $x = 8^{-1}, 8^{-2}, 8^{-3},...$.
\[ f(x) = \sqrt{x^2 + 1} - 1 \]
\[ g(x) = \frac{x^2}{\sqrt{x^2 + 1} + 1} \]
\subsection{Wyniki}
\begin{center}
  \begin{tabular}{c|c|c}
    x & f(x) & g(x)\\
    \hline
    $8^{-1}$ & 0.0077822185373186414 & 0.0077822185373187065 \\
    $8^{-2}$ & 0.00012206286282867573 & 0.00012206286282875901 \\
    $8^{-3}$ & $1.9073468138230965 * 10^{-6}$ & $1.907346813826566 * 10^{-6}$ \\
    $8^{-4}$ & $2.9802321943606103 * 10^{-8}$ & $2.9802321943606116 * 10^{-8}$ \\
    $8^{-5}$ & $4.656612873077393 * 10^{-10}$ & $4.6566128719931904 * 10^{-10}$ \\
    $8^{-6}$ & $7.275957614183426 * 10^{-6}$12 & $7.275957614156956 * 10^{-12}$ \\
    $8^{-7}$ & $1.1368683772161603 * 10^{-13}$ & $1.1368683772160957 * 10^{-13}$
  \end{tabular}
\end{center}
\subsection{Wnioski}



\section{Zadanie 7 - Pochodna oraz różnica błędu}
\subsection{Problem}
Należało porównać przyblizoną wartość pochodnej funkcji f  w punkcie $x_{0} = 1$ z jej rzeczywistą wartością. Gdzie $f(x) = \sin{x} + \cos{3x}$.

 Błąd do obliczenia:
\[|f'(x_{0}) - \widetilde{f}'(x_{0})|\]

Natomiast przybliżenie jest liczone ze wzoru:
\[\widetilde{f}'(x_{0}) = \frac{f(x_{0} + h) - f(x_{0})}{h}\]

Przy obliczeniach należało używać arytmetyki Float64 oraz $h = 2^{-n}$ , gdzie n = 0,1,2,..52
\subsection{Wyniki}
Wyliczony bład z powyższego wzoru dla poszczególnych n oraz pokazanie zmiany h
\begin{center}
  \begin{tabular}{c|c|c}
    n & h & błąd  \\
    \hline
    0 & 1.0 & 1.9010469435800585\\
    1 & 0.5 & 1.753499116243109\\
    2 & 0.25 & 0.9908448135457593\\
    \vdots \qquad & \vdots \qquad & \vdots \qquad \\
    27 & $7.450580596923828 * 10^{-9}$ & $3.460517827846843 * 10^{-8}$\\
    28 & $3.725290298461914 * 10^{-9}$ & $4.802855890773117 * 10^{-9}$\\
    29 & $1.862645149230957 * 10^{-9}$ & $5.480178888461751 * 10^{-8}$\\
    \vdots \qquad & \vdots \qquad & \vdots \qquad \\
    50 & $8.881784197001252 * 10^{-16}$ & 0.11694228168853815\\
    51 & $4.440892098500626 * 10^{-16}$ & 0.11694228168853815\\
    52 & $2.220446049250313 * 10^{-16}$ & 0.6169422816885382
  \end{tabular}
\end{center}
\subsection{Wnioski}

\end{document}
