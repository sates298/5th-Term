\documentclass[11pt, a4paper]{article}
\usepackage[T1]{fontenc}
\usepackage[utf8]{inputenc}
\usepackage[MeX]{polski}
\usepackage[polish]{babel}
\usepackage{enumerate}
\usepackage{float}
\usepackage{geometry}
\usepackage{amsmath}
\usepackage[linesnumbered,ruled]{algorithm2e}
\usepackage{graphicx}
\usepackage{caption}


\graphicspath{{../plots/}}

\geometry{top=1.5cm, bottom=1.5cm, right=1.5cm, left=1.5cm}


\title{Obliczenia naukowe\\Lista4}
\author{Stanisław Woźniak}
\date{}

\begin{document}
    \maketitle
    \section{Zadanie 1.}
    \subsection{Ilorazy różnicowe}
    \subsection{Opis}
    \subsection{Pseudokod}
    \section{Zadanie 2.}
    \subsection{Wielomian interpolacyjny Newtona}
    \subsection{Opis}
    \subsection{Pseudokod}
    \section{Zadanie 3.}
    \subsection{Postać naturalna wielomianu interpolacyjnego Newtona}
    \subsection{Opis}
    \subsection{Pseudokod}
    \section{Zadanie 4.}
    \subsection{Rysowanie wykresów}
    \subsection{Opis}
    \subsection{Pseudokod}
    \section{Zadanie 5.}
    \subsection{Problem}
    \subsection{Wyniki}
    \subsection{Wnioski}
    \section{Zadanie 6.}
    \subsection{Problem}
    \subsection{Wyniki}
    \subsection{Wnioski}
\end{document}