\documentclass[11pt, a4paper]{article}
\usepackage[T1]{fontenc}
\usepackage[utf8]{inputenc}
\usepackage[MeX]{polski}
\usepackage[polish]{babel}
\usepackage{enumerate}
\usepackage{float}
\usepackage{geometry}
\usepackage{amsmath}
\usepackage[linesnumbered,ruled]{algorithm2e}


\geometry{top=1.5cm, bottom=1.5cm, right=1.5cm, left=1.5cm}


\title{Obliczenia naukowe\\Lista3}
\author{Stanisław Woźniak}
\date{}

\begin{document}
    \maketitle
    \section{Zadanie 1.}
    \subsection{Metoda Bisekcji}
    \subsection{Opis}
    Jedna z metod znajdowania miejsca zerowego funkcji ciągłej na podanym przedziale.\\
    Metoda polega na połowieniu danego przedziału z każdą iteracją do momentu znalezienia szukanego $x$ z dokładnością do $\delta$ lub do momentu gdy $|f(x)| < \epsilon$
    Z każdą iteracją jest definiowany nowy przedział. Jeden koniec ma w połowie poprzedniego przedziału. Natomiast drugi jest wybierami z dwóch poprzednich z warunkiem, że znak wartości funkcji na krańcach przedziału jest przeciwny.\\
    Warunki początkowe:
    \begin{enumerate}
        \item Badana funkcja posiada miejsce zerowe.
        \item Badan funkcja jest ciągła na przedziale [a,b]
        \item Na krańcach przedziału wartość funkcji musi mieć przeciwne znaki.
    \end{enumerate}
    \subsection{Pseudokod}
    \begin{algorithm}[H]
        \SetKwInOut{Input}{Input}
        \SetKwInOut{Output}{Output}

        \underline{Metoda Bisekcji} $(f, a,b, \delta,\epsilon)$\;
        \Input{f - funkcja, [a,b] - przedział, $\delta$ - dokładność $x_{0}$, $\epsilon$ - dokładność $f(x_{0})$}
        \Output{k, $x_{0}$, $f(x_{0})$}
        $x_{0} \leftarrow 0;k \leftarrow 0$\;
        \While{$|a - b| > \delta$}
        {
            $k++$\;
            \uIf{$f(a)*f(b) < 0$}
            {
                $x_{0} \leftarrow a + \frac{b-a}{2}$\;
            }
            \Else{
                return ''Error: Funkcja nie zmienia znaku w przedziale''\;
            }

            \If{$|f(x_{0})| < \epsilon$}
            {
                return $(k, x_{0}, f(x_{0}))$\;
            }

            \uIf{$f(x_{0})*f(a) < 0$}
            {
                $b \leftarrow x_{0}$\;
            }\ElseIf{$f(x_{0})*f(b) < 0$}
            {
                $a \leftarrow x_{0}$\;
            }
        }
        return $(k, x_{0}, f(x_{0}))$\;
        \caption{Metoda Bisekcji}
    \end{algorithm}
    
    \section{Zadanie 2.}
    \subsection{Metoda Newtona (stycznych)}
    \subsection{Opis}
    Jest to algorytm iteracyjny przybliżajacy pierwiastek funkcji. Metoda polega na wyprowadzaniu stycznych z wybranego punktu $f(x_{0})$. Punkt przecięcia stworzonej stycznej z osią OX jest szukanym miejsciem zerowym. Jeśli się okarze, ze przybliżenie jest zbyt mało dokładne czynność jest powtarzana gdzie do $x_{0}$ przypisuje się wyznaczone poprzednio miejsce zerowy. 
    \subsection{Pseudokod}
    \begin{algorithm}[H]
        \SetKwInOut{Input}{Input}
        \SetKwInOut{Output}{Output}

        \underline{Metoda Bisekcji} $(f, f',x_{0}, \delta,\epsilon, maxit)$\;
        \Input{f - funkcja, f' - pochodna funkcji, $x_{0}$ - przybliżenie początkowe, $\delta$ - dokładność $x_{0}$, $\epsilon$ - dokładność $f(x_{0})$, maxit - maksymalna liczba iteracji}
        \Output{k, $x_{0}$, $f(x_{0})$}
        $k \leftarrow 0; x_{1} \leftarrow x_{0} - 1; v \leftarrow f(x_{0})$\;
        \While{$|x_{1} - x_{0}|> \delta$}
        {
            $k++$\;
            \If{$k > maxit$}
            {
                return ''Error: Przekroczenie liczby iteracji''\;
            }
            \If{$|f'(x_{0})| < \epsilon$}
            {
                return ''Error: Pochodna bliska zeru''\;
            }

            $x_{1} \leftarrow x_{0}; x_{0} \leftarrow x_{0} - \frac{v}{f'(x_{0})}; v \leftarrow f(x_{0})$\;

            \If{$|v| < \epsilon$}
            {
                return $(k, x_{0}, v)$\;
            }
        }
        return $(k, x_{0}, v)$\;
        \caption{Metoda Newtona}
    \end{algorithm}
    \section{Zadanie 3.}
    \subsection{Metoda Siecznych}
    \subsection{Opis}
    Metoda wyznaczania przybliżenia miejsca zerowego funkcji. W tym algorytmi przyjmuje się, że podana funkcja jest ciągła, oraz na dostatecznie małym odcinku w przybliżeniu zmienia się w sposób liniowy. To założenie pozwala nam zastąpić dany fragment wykresu funkcji sieczną. Punkt przecięcia siecznej z osią OX jest szukanym przybliżeniem miejsca zerowego. Jeśli przybliżenie nie jest wystarczająco dokładne, czynność zostaje powtarzana przyjmując punkt wyliczony w poprzedniej iteracji jako koniec siecznej.
    
    Warunek powodzenia:\\
    
    $$\bigwedge_{n} (f(x_{n})f(x_{n-1}) < 0) $$
    
    \subsection{Pseudokod}
    \begin{algorithm}[H]
        \SetKwInOut{Input}{Input}
        \SetKwInOut{Output}{Output}

        \underline{Metoda Siecznych} $(f, x_{0},x_{1}, \delta,\epsilon, maxit)$\;
        \Input{f - funkcja, $x_{0}, x_{1}$ - przybliżenia początkowe, $\delta$ - dokładność $x_{0}$, $\epsilon$ - dokładność $f(x_{0})$, maxit - maksymalna liczba iteracji}
        \Output{k, $x_{0}$, $f(x_{0})$}
        $fa \leftarrow f(x_{0}); fb \leftarrow f(x_{1}); k \leftarrow 0$\;
        \While{$|x_{1} - x_{0}| > \delta$}
        {
            $k++$\;
            \If{$k > maxit$}
            {
                return ''Error: Przekroczenie liczby iteracji''\;
            }

            \If{$|fa| > |fb|$}
            {
                $x_{0} \leftrightarrow x_{1}; fa \leftrightarrow fb$\;
            }

            $s \leftarrow \frac{(x_{0} - x_{1})}{fb - fa}$\;
            $x_{1} \leftarrow x_{0}; fb \leftarrow fa$\;
            $x_{0} \leftarrow x_{0} - fa*s; fa \leftarrow f(x_{0})$\;
            \If{$|fa| < \epsilon$}
            {
                return $(k, x_{0}, fa)$\;
            }
        }
        return $(k, x_{0}, fa)$\;
        \caption{Metoda Siecznych}
    \end{algorithm}
    \section{Zadanie 4.}
    \subsection{Problem}
    Problem polegał na wyznaczeniu pierwiastka równania przy użyciu zaimplementowanych metod w poprzednich zadaniach.
    
    Równanie:
    \[\sin{x} - (\frac{1}{2}x)^{2} = 0 \]

    Dla każdej z metod zostały użyte te same dokładności $\delta$ oraz $\epsilon$ równe $\frac{1}{2} * 10^{-5}$.\\
    Jednakże początkowe przybliżenia oraz przedziały dla każdej z nich zostały zdefiniowane inne.
    \begin{enumerate}
        \item Metoda bisekcji - przedział początkowy $[1.5, 2]$
        \item Metoda Newtona (Metoda stycznych) - przybliżenie początkowe $x_{0} = 1.5$
        \item Metoda siecznych - przybliżenia poczatkowe $x_{0} = 1$, $x_{1} = 2$
    \end{enumerate}
    \subsection{Wyniki}
    Na potrzeby rozwiązania równania danymi metodami lewa strona równania została uznana za funkcję f
    Oznaczenia:\\
    r - znalezione miejsce zerowe z dokładnością do $\delta$ \\
    v - wartość funkcji w punkcie r z dokładnością do $\epsilon$ \\
    it - liczba wykonanych iteracji potrzebnych do znalezienia pierwiastka \\
    err - powiadomienie o błędzie

    Prawidłowy wynik: r = 0 , v = 0.
    \begin{center}
        \begin{tabular}{c|c|c|c|c}
            metoda & r & v & it & err\\
            \hline
            bisekcji & 1.9337539672851562 & -2.7027680138402843e-7 & 16 & Brak błędu\\
            stycznych & 1.933749984135789 & 4.995107540040067e-6 & 13 & Brak błędu\\
            siecznych & 1.933753644474301 & 1.564525129449379e-7 & 4 & Brak błędu\\
        \end{tabular}
    \end{center}
    \subsection{Wnioski}

    \section{Zadanie 5.}
    \subsection{Problem}
    Zadanym problemem było znaleźć punkty przecięcia się wykresów dwóch funkcji.
    \begin{enumerate}
        \item $f(x) = 3x$
        \item $g(x) = e^{x}$
    \end{enumerate}
    Dokładność obliczeniowe wynosiły: $\delta = \epsilon = 10^{-4}$\\
    Aby znaleźć odpowiednie punkty przecięcia należało stworzyć funkcję pomocniczą h.
    \[h(x) = f(x) - g(x)\]
    Ten sposób zapewniał znalezienie punkty przeciecia wykresów funkcji f oraz g metodą znalezienia miejsc zerowych funkcji h.\\
    Do znalezienia rozwiązania należało użyć metodę bisekcji.
    \subsection{Wyniki}
    [a,b] - przedział początkowy \\
    r - znalezione miejsce zerowe z dokładnością do $\delta$ \\
    v - wartość funkcji w punkcie r z dokładnością do $\epsilon$ \\
    it - liczba wykonanych iteracji potrzebnych do znalezienia pierwiastka \\
    err - powiadomienie o błędzie

    Prawidłowy wynik (przybliżony): $r_{1}$ = 0.619061, $r_{2}$ = 1.51213.
    \begin{center}
        \begin{tabular}{c|c|c|c|c}
            [a, b] & r & v & it & err\\
            \hline
            $[0.5, 0.7]$ & 0.619140625 & 9.066320343276146e-5 & 9 & Brak błędu\\
            $[-1, 1]$ & 0.619140625 & 9.066320343276146e-5 & 10 & Brak błędu\\
            $[0.0, 1.5]$ & 0.61907958984375 & 2.091677592419572e-5 & 13 & Brak błędu\\
            $[1, 2]$ & 1.5120849609375 & 7.618578602741621e-5 & 13 & Brak błędu\\
            $[1.5, 1.7]$ & 1.512109375 & 3.868007140983565e-5 & 9 & Brak błędu\\
            $[-1, 2]$ & - & - & - & Funkcja nie zmienia znaku na przedziale [a,b]\\
            $[-10, 0]$ & - & - & - & Funkcja nie zmienia znaku na przedziale [a,b]\\
            $[1, 1.5]$ & - & - & - & Funkcja nie zmienia znaku na przedziale [a,b]\\
            $[2, 10]$ & - & - & - & Funkcja nie zmienia znaku na przedziale [a,b]
        \end{tabular}
    \end{center}
    \subsection{Wnioski}

    \section{Zadanie 6.}
    \subsection{Problem}
    Należało znaleźć miejsca zerowe trzema metodami (bisekcji, stycznych oraz siecznych) dwóch funkcji:
    \begin{enumerate}
        \item $f(x) = e^{1-x} - 1$ (prawidłowe rozwiązanie: x = 1)
        \item $g(x) = xe^{-x}$ (prawidłowe rozwiązanie: x = 0)
    \end{enumerate}
    Obliczenia należało wykonać z dokłądnością $\delta = \epsilon = 10^{-5}$\\
    Także zadaniem było dobrać odpowiedni przedział i przybliżenia początkowe.
    \subsection{Wyniki}
    r - znalezione miejsce zerowe z dokładnością do $\delta$ \\
    v - wartość funkcji w punkcie r z dokładnością do $\epsilon$ \\
    it - liczba wykonanych iteracji potrzebnych do znalezienia pierwiastka \\
    err - powiadomienie o błędzie

    \begin{center}
        \begin{tabular}{|c|c|c|c|c|c|}
            \hline
            \multicolumn{6}{|c|}{Metoda bisekcji}\\
            \hline
            funkcja & [a,b] & r & v & it & err\\
            \hline
            f & $[0.5, 2]$ & 0.9999923706054688 & 7.629423635080457e-6 & 16 & Brak błędu\\
            f & $[-0.1, 2]$ & 1.0000038146972656 & -3.814689989667386e-6 & 17 & Brak błędu\\
            f & $[0.0, 2]$ & 1.0 & 0.0 & 1 & Brak błędu\\
            f & $[-100, 100]$ & 0.9999990463256836 & 9.536747711536009e-7 & 23 & Brak błędu\\
            f & $[0.99999, 20]$ & 1.0000081198215485 & -8.119788582838794e-6 & 20 & Brak błędu\\
            \hline
            g & $[-0.5, 1]$ & -7.62939453125e-6 & -7.629452739132958e-6 & 16 & Brak błędu\\
            g & $[-0.5, 0.5]$ & 0.0 & 0.0 & 1 & Brak błędu\\
            g & $[-0.1, 4]$ & 3.8146972656107834e-6 & 3.8146827137233106e-6 & 17 & Brak błędu\\
            g & $[-100, 101]$ & 4.410743713378906e-6 & 4.410724258761706e-6 & 23 & Brak błędu\\
            \hline
        \end{tabular}
    \end{center}
    
    \begin{center}
        \begin{tabular}{|c|c|c|c|c|c|}
            \hline
            \multicolumn{6}{|c|}{Metoda Newtona}\\
            \hline
            funkcja & x0 & r & v & it & err\\
            \hline
            f & 4 & 0.9999999995278234 & 4.721765201054495e-10 & 21 & Brak błędu\\
            f & 1 & 0.9999999810061002 & 1.8993900008368314e-8 & 5 & Brak błędu\\
            f & 11 & NaN & NaN & 2 & Wyjście poza zakres\\
            f & 101 & NaN & NaN & 1 & Pochodna bliska zeru\\
            \hline
            g & -1 & -3.0642493416461764e-7 & -3.0642502806087233e-7 & 5 & Brak błędu\\
            g & 1 & NaN & NaN & 1 & Pochodna bliska zeru\\
            g & 2 & 14.398662765680003 & 8.036415344217211e-6 & 10 & Brak błędu\\
            g & 11 & 14.272123938290518 & 9.040322779745372e-6 & 3 & Brak błędu\\
            g & 101 & NaN & NaN & 1 & Pochodna bliska zeru\\
            \hline
        \end{tabular}
    \end{center}

    \begin{center}
        \begin{tabular}{|c|c|c|c|c|c|c|}
            \hline
            \multicolumn{7}{|c|}{Metoda siecznych}\\
            \hline
            funkcja & x1 & x2 & r & v & it & err\\
            \hline
            f & 0.5 & 2 & 1.000000014307199 & -1.4307198870078253e-8 & 6 & Brak błędu\\
            f & -0.1 & 2 & 1.0000032272298756 & -3.227224668056472e-6 & 6 & Brak błędu\\
            f & 0 & 2 & 1.0000017597132702 & -1.7597117218937086e-6 & 6 & Brak błędu\\
            f & -100 & 100 & 100.0 & -1.0 & 1 & Brak błędu\\
            f & 0.99999 & 20 & 1.0000000009000212 & -9.000211687038018e-10 & 2 & Brak błędu\\
            \hline
            g & -1 & 0.5 & -1.1737426154042664e-6 & -1.1737439930768023e-6 & 7 & Brak błędu\\
            g & -0.5 & 0.5 & 5.38073548562323e-6 & 5.380706533386756e-6 & 6 & Brak błędu\\
            g & -0.1 & 4 & 14.32970132001514 & 8.568936563065177e-6 & 14 & Brak błędu\\
            g & -100 & 100 & 100.0 & 3.7200759760208363e-42 & 1 & Brak błędu\\
            \hline
        \end{tabular}
    \end{center}
    \subsection{Wnioski}

\end{document}